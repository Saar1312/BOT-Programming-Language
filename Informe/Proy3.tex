\documentclass{article} 
\usepackage{graphicx}
\usepackage[spanish]{babel}
\usepackage[autostyle]{csquotes}
\usepackage{tabto}
\usepackage{booktabs}

\renewcommand{\contentsname}{Contenido}
\renewcommand{\figurename}{Im\'agen}


% \newcommand{\itab}[1]{\hspace{0em}\rlap{#1}}
% \newcommand{\tab}[1]{\hspace{.05\textwidth}\rlap{#1}}

\title{Proyecto 3\\Traductores} 
\author{Samuel Arleo\and Sergio Ter\'an } 
\date{06 Marzo, 2016}

\begin{document}

	%Portada
		\pagenumbering{gobble}
		\begin{center}
		 	\pagestyle{headings} \small
		 	\includegraphics[scale=0.4]{USB.png}\\[0.3cm]
			Universidad Sim\'on Bolivar\\[0.1cm]
		 	Departamento de Computaci\'on y Tecnolog\'ia de la Informaci\'on\\[0.1cm]
		 	CI3725 - Traductores e Interpretadores\\[0.1cm]
		 	Enero-Marzo 2016\\[0.1cm]

			\vspace{15em}
			\LARGE Proyecto 3\\
			\vspace{0.5em}
			\Large Traductores\\
			\vspace{0.5em}
			\large Samuel Arleo \-\hspace{5em}Sergio Ter\'an\\
			\vspace{0.5em}
			\large 10-10969 \-\hspace{7em}11-11020\\
			\vspace{15em}
			07 de Marzo, 2016
		\end{center}

	%Seccion sobre la implementacion
		\newpage
		\pagenumbering{arabic}
		\section{Formulaci\'on e Implementaci\'on} 

	%Seccion de respuestas teorico practicas
		\newpage
		\section{Revisi\'on Te\'orico-Practica}
			\begin{enumerate}
				%Pregunta 1
				\large \bf \item[] Pregunta 1
				\normalsize \mdseries
					\begin{enumerate}
					%Parte (a)
					\item[(a)] \
						\begin{enumerate}
							\item[(a.1)] 
								$G_{1}$ = (\{S\},\{a\},\{S $\longrightarrow$ Sa,S $\longrightarrow$ $\lambda$\},S)\\
								\\
								Determinemos si la gram\'atica\\
								\begin{tabbing}
								\hspace*{2cm} \= \hspace*{0.6cm} \= \hspace*{3cm} \kill
									%p_inicio
									\>S \' $\longrightarrow$\> Sa\\
									\>S \' $\longrightarrow$\> $\lambda$\\
								\end{tabbing}
								Es LR(1) y construyamos su analizador sint\'actico. 
								Comenzamos por aumentar la gramatica con el s\'imbolo $S^{\prime}$ y agregando el s\'imbolo \$ al final de la primera entrega. A dem\'as enumeramos las producciones.
								\begin{tabbing}
								\hspace*{1cm} \= \hspace*{1cm} \= \hspace*{1cm} \= \hspace*{0.6cm} \= \hspace*{0.6cm} \= \hspace*{3cm} \kill
									%item 0
									\> (i)\>  $S^{\prime}$	\> $\longrightarrow$\> 	S\$ 	\\
									\> (ii)\>  $S$	\> $\longrightarrow$\> 	Sa 	\\
									\> (iii)\>  $S$	\> $\longrightarrow$\>  $\lambda$ 	\\
								\end{tabbing}
								Construimos los conjuntos FIRST y FOLLOW para los simbolos no terminales:
								\begin{quotation}
									FIRST($S^{\prime}$) = \{ $\lambda$ , a \}
								\end{quotation}
								\begin{quotation}
									FIRST($S$) = \{ $\lambda$ , a \}
								\end{quotation}
								\begin{quotation}
									FOLLOW($S^{\prime}$) = \{ \$ \}
								\end{quotation}
								\begin{quotation}
									FOLLOW($S$) = \{ a, \$ \}
								\end{quotation}
								El conjunto de clauduras nos queda:\\
								\begin{tabbing}
								 \hspace*{1cm} \= \hspace*{1cm} \= \hspace*{0.6cm} \= \hspace*{0.6cm} \= \hspace*{3cm} \kill
									%item 0
									\> $I_{0}$	\' : 	\> $S^{\prime}$	\> $\longrightarrow$\> 	$\cdot$S\$ 	\\
									\>		\'  	\> $S$ 			\> $\longrightarrow$\> 	$\cdot$Sa 	\\
									\>		\'  	\> $S$ 			\> $\longrightarrow$\> 	$\cdot$ 	\\
								
									%item 1
									\>$I_{1}$	\' : 	\> $S^{\prime}$ \> $\longrightarrow$\> 	S$\cdot$\$	\\
									\>		\'  	\> $S$ 			\> $\longrightarrow$\> 	S$\cdot$a	\\
									
									%item 2
									\>$I_{2}$	\' : 	\> $S^{\prime}$ \> $\longrightarrow$\> 	S\$$\cdot$	\\
									
									%item 3
									\>$I_{3}$	\' : 	\> $S$ 			\> $\longrightarrow$\> 	Sa$\cdot$	\\
								\end{tabbing}
								Construimos el automata de prefijos viables, que nos queda de la forma:\\
				 				
							  		\includegraphics[ scale=0.2]{grafo1.png}
							  	\\Ahora podemos constriur la tabla de parsing $SLR(1)$:
							  	\\
				 				\begin{tabular}{l | c c | c c}
									\toprule
									&\multicolumn{2}{|c|}{Acciones} &\multicolumn{2}{|c}{Goto}\\
									\midrule[0.5mm]
									& a & \$ & $S^{\prime}$ & S \\
									\midrule[0.5mm]
										$I_{0}$ & r(iii) & r(iii) &  & 1 \\
									\hline
										$I_{1}$ & s(3) & s(2) &  &  \\
									\hline
										$I_{2}$ &  & $acc$ &  &  \\
									\hline
										$I_{3}$ & r(ii) & r(ii) &  &  \\
									\bottomrule
								\end{tabular}
							\\\\\
							\item[(a.2)] 
								$G_{1}$ = (\{S\},\{a\},\{S $\longrightarrow$ aS,S $\longrightarrow$ $\lambda$\},S)\\
								\\
								Determinemos si la gram\'atica\\
								\begin{tabbing}
								\hspace*{2cm} \= \hspace*{0.6cm} \= \hspace*{3cm} \kill
									%p_inicio
									\>S \' $\longrightarrow$\> aS\\
									\>S \' $\longrightarrow$\> $\lambda$\\
								\end{tabbing}
								Es LR(1) y construyamos su analizador sint\'actico. 
								Comenzamos por aumentar la gramatica con el s\'imbolo $S^{\prime}$ y agregando el s\'imbolo \$ al final de la primera entrega. A dem\'as enumeramos las producciones.
								\begin{tabbing}
								\hspace*{1cm} \= \hspace*{1cm} \= \hspace*{1cm} \= \hspace*{0.6cm} \= \hspace*{0.6cm} \= \hspace*{3cm} \kill
									%item 0
									\> (i)\>  $S^{\prime}$	\> $\longrightarrow$\> 	S\$ 	\\
									\> (ii)\>  $S$	\> $\longrightarrow$\> 	aS 	\\
									\> (iii)\>  $S$	\> $\longrightarrow$\>  $\lambda$ 	\\
								\end{tabbing}
								Construimos los conjuntos FIRST y FOLLOW para los simbolos no terminales:
								\begin{quotation}
									FIRST($S^{\prime}$) =  FIRST($S$) = \{ $\lambda$ , a \}
								\end{quotation}

								\begin{quotation}
									FOLLOW($S^{\prime}$) =  FOLLOW($S$) = \{ \$ \}
								\end{quotation}

								El conjunto de clauduras nos queda:\\
								\begin{tabbing}
								 \hspace*{1cm} \= \hspace*{1cm} \= \hspace*{0.6cm} \= \hspace*{0.6cm} \= \hspace*{3cm} \kill
									%item 0
									\> $I_{0}$	\' : 	\> $S^{\prime}$	\> $\longrightarrow$\> 	$\cdot$S\$ 	\\
									\>		\'  	\> $S$ 			\> $\longrightarrow$\> 	$\cdot$aS 	\\
									\>		\'  	\> $S$ 			\> $\longrightarrow$\> 	$\cdot$ 	\\
								
									%item 1
									\>$I_{1}$	\' : 	\> $S^{\prime}$ \> $\longrightarrow$\> 	S$\cdot$\$	\\
									
									%item 2
									\>$I_{2}$	\' : 	\> $S$ \> $\longrightarrow$\> 	a$\cdot$S	\\
									\>			\'  	\> $S$ \> $\longrightarrow$\> 	$\cdot$aS	\\
									\>			\'  	\> $S$ \> $\longrightarrow$\> 	$\cdot$	\\
									
									%item 3
									\>$I_{3}$	\' : 	\> $S$ 			\> $\longrightarrow$\> 	aS$\cdot$	\\

									%item 4
									\>$I_{4}$	\' : 	\> $S^{\prime}$	\> $\longrightarrow$\> 	S\$ $\cdot$	\\
								\end{tabbing}
								Construimos el automata de prefijos viables, que nos queda de la forma:\\
				 				
							  		\includegraphics[ scale=0.2]{grafo2.png}
							  	\\Ahora podemos constriur la tabla de parsing $SLR(1)$:
							  	\\
				 				\begin{tabular}{l | c c | c c}
									\toprule
									&\multicolumn{2}{|c|}{Acciones} &\multicolumn{2}{|c}{Goto}\\
									% &Acciones& &GoTo\\
									\midrule[0.5mm]
									& a & \$ & $S^{\prime}$ & S \\
									\midrule[0.5mm]
										$I_{0}$ & s(2) & r(iii) &  & 1 \\
									\hline
										$I_{1}$ &  & s(4) &  &  \\
									\hline
										$I_{2}$ & s(2) & r(iii) &  & 3 \\
									\hline
										$I_{3}$ &  & r(ii) &  &  \\
									\hline
										$I_{4}$ &  & $acc$ &  &  \\
									\bottomrule
								\end{tabular}


						\end{enumerate}
					%Parte (b)
					\item[(b)] Comparando las tablas de $G1_{i}$ y $G1_{d}$ vemos que la tabla de parsing de $G1_{i}$ contiene menos filas que $G1_{d}$. 
					\\\\
					En la pila, $G1_{i}$, consume un maximo de 3 items en la pila, en el caso especifico de la frase $aaa$ la pila  para $G1_{d}$ puede llegar a tener hasta 4 items, por lo que podemos ver que $G1_{i}$ es mas eficiente en cuanto al consumo de memoria por parte de la pila.
					\\\\
					a dem\'as vemos que para reconocer las frases, la cantidad de movimientos realizados por ambos es de $2*n+3$ po lo que el tiempo de ejecucion es de orden $O(n)$ donde $n$ es la cantidad de `a' en la frase.
					\\\\ 

					
					\end{enumerate}
				\large \bf \item[] Pregunta 2
				\normalsize \mdseries
					\begin{enumerate}
					%Parte (a)
					\item[(a)]								
					\begin{tabbing}
								\hspace*{1cm} \= \hspace*{1cm} \= \hspace*{1cm} \= \hspace*{0.6cm} \= \hspace*{0.6cm} \= \hspace*{3cm} \kill
									\> (i)\>  $S^{\prime}$	\> $\longrightarrow$\> 	$Instr$ 	\\
									\> (ii)\>  $Instr$	\> $\longrightarrow$\> 	$Instr$ ; $Instr$ 	\\
									\> (iii)\>  $Instr$	\> $\longrightarrow$\>  IS 	\\
					\end{tabbing}
					\begin{quotation}
						FIRST($S^{\prime}$) =  FIRST($Instr$) = \{ IS \}
					\end{quotation}		
					\begin{quotation}
						FOLLOW($S^{\prime}$) \{ \$ \}
					\end{quotation}
					\begin{quotation}
						FOLLOW($Instr$) \{ ;,\$ , IS \}
					\end{quotation}						
					\ \ 
					\begin{tabbing}
					 \hspace*{1cm} \= \hspace*{1cm} \= \hspace*{0.8cm} \= \hspace*{0.6cm} \= \hspace*{3cm} \kill
						%item 0
						\> $I_{0}$	\' : 	\> $S^{\prime}$		\> $\longrightarrow$\> 	$\cdot Instr$ \$	\\
						\>			\'  	\> $Instr$ 			\> $\longrightarrow$\> 	$\cdot Insrr$ ; $Instr$	\\
						\>			\'  	\> $Instr$ 			\> $\longrightarrow$\> 	$\cdot$ IS	\\


						%item 1
						\> $I_{1}$	\' : 	\> $S^{\prime}$		\> $\longrightarrow$\> 	$Instr \cdot$ \$	\\
						\>			\'  	\> $Instr$ 			\> $\longrightarrow$\> 	$Instr \cdot$ ; $Instr$	\\

						%item 2
						\>$I_{2}$	\' : 	\> $Instr$ 			\> $\longrightarrow$\> 	IS$\cdot$	\\

						%item 3
						\>$I_{3}$	\' : 	\> $S^{\prime}$		\> $\longrightarrow$\> 	$Instr$\$$\cdot$	\\

						%item 4
						\>$I_{4}$	\' : 	\> $Instr$			\> $\longrightarrow$\> 	$Instr$ ; $\cdot$ $Instr$	\\
						\>			\'  	\> $Instr$ 			\> $\longrightarrow$\> 	$\cdot Instr$ ; $Instr$	\\
						\>			\'  	\> $Instr$ 			\> $\longrightarrow$\> 	$\cdot$IS	\\

						%item 5
						\>$I_{5}$	\' : 	\> $Instr$			\> $\longrightarrow$\> 	$Instr$ ; $Instr$ $\cdot$	\\
						\>			\'  	\> $Instr$ 			\> $\longrightarrow$\> 	$Instr \cdot$ ; $Instr$	\\
					\end{tabbing}

					En la regla $I_{5}$ vemos que existe un conflicto 
					\
					\\
					%Parte (b)
					\item[(b)]Este conflicto, del tipo $shift/reduce$, lo intentaremos solucionar usando el algoritmo de SLR(1), apoyandonos con los FIRST y FOLLOW que ya hemos calculado.
	 				\\\\\
	 				\begin{tabular}{l | c c c | c c}
						\toprule
						&\multicolumn{3}{|c|}{Acciones} &\multicolumn{2}{|c}{Goto}\\
						% &Acciones& &GoTo\\
						\midrule[0.5mm]
						& ; & IS & \$ & $S^{\prime}$ & $Instr$ \\
						\midrule[0.5mm]
							$I_{0}$ &  & s(2) &  &  & 1 \\
						\hline
							$I_{1}$ & s(4) &  & s(3) &  \\
						\hline
							$I_{2}$ & r(iii) & r(iii) & r(iii) &  \\
						\hline
							$I_{3}$ &  &  & $acc$ &  \\
						\hline
							$I_{4}$ &  & s(2) &  &  \\
						\hline
							$I_{4}$ & \textbf{s(4)/r(ii)} & r(ii) & r(ii) &  \\
						\bottomrule
					\end{tabular}
					\ 
					\\\\
					%Parte (c)
					\item[(c)]
					Secuencia de reconocimiento para la frase IS;IS;IS, dando prioridad al $shift$ en el conflicto $shift/reduce$
					\ 
					\\\\
					\begin{tabular}{|c|c|c|c|}
						\hline
						Pila & Entrada & Acci\'on & Salida\\	
						\hline
						$I_{0}$ & IS;IS;IS\$ & s(2) &
						\\
						\hline
						$I_{2}$
						$I_{0}$ & ;IS;IS\$ & r(iii) &
						(iii)\\
						\hline
						$I_{1}$
						$I_{0}$ & ;IS;IS\$ & s(4) &
						(iii)\\
						\hline
						$I_{4}$
						$I_{1}$
						$I_{0}$ & IS;IS\$ & s(2) &
						(iii)\\
						\hline
						$I_{2}$
						$I_{4}$
						$I_{1}$
						$I_{0}$ & ;IS\$ & r(iii) &
						(iii),
						(iii)\\
						\hline
						$I_{5}$
						$I_{4}$
						$I_{1}$
						$I_{0}$ & ;IS\$ & s(4) &
						(iii),
						(iii)\\
						\hline
						$I_{4}$
						$I_{5}$
						$I_{4}$
						$I_{1}$
						$I_{0}$ & IS\$ & s(2) &
						(iii),
						(iii)\\
						\hline
						$I_{2}$
						$I_{4}$
						$I_{5}$
						$I_{4}$
						$I_{1}$
						$I_{0}$ & \$ & r(iii) &
						(iii),
						(iii),
						(iii)\\
						\hline
						$I_{5}$
						$I_{4}$
						$I_{5}$
						$I_{4}$
						$I_{1}$
						$I_{0}$ & \$ & r(ii) &
						(ii),
						(iii),
						(iii),
						(iii)\\
						\hline
						$I_{5}$
						$I_{4}$
						$I_{1}$
						$I_{0}$ & \$ & r(ii) &
						(ii),
						(ii),
						(iii),
						(iii),
						(iii),\\
						\hline
						$I_{1}$
						$I_{0}$ & \$ & s(3) &
						(ii),
						(ii),
						(iii),
						(iii),
						(iii),\\
						\hline
						$I_{3}$
						$I_{1}$
						$I_{0}$ & \$ & $acc$ &
						(ii),
						(ii),
						(iii),
						(iii),
						(iii),\\
						\hline

					\end{tabular}
					% \newpage
					\\
					\\
					\\Secuencia de reconocimiento para la frase IS;IS;IS, dando prioridad al $reduce$ en el conflicto $shift/reduce$\\

					\begin{tabular}{|c|c|c|c|}
						\hline
						Pila & Entrada & Acci\'on & Salida\\	
						\hline
						$I_{0}$ & IS;IS;IS\$ & s(2) &
						\\
						\hline
						$I_{2}$
						$I_{0}$ & ;IS;IS\$ & r(iii) & 
						(iii)\\
						\hline
						$I_{1}$
						$I_{0}$ & ;IS;IS\$ & s(4) & 
						(iii)\\
						\hline
						$I_{4}$
						$I_{1}$
						$I_{0}$ & IS;IS\$ & s(2) & 
						(iii)\\
						\hline
						$I_{2}$
						$I_{4}$
						$I_{1}$
						$I_{0}$ & ;IS\$ & r(iii) & 
						(iii),
						(iii) \\
						\hline
						$I_{5}$
						$I_{4}$
						$I_{1}$
						$I_{0}$ & ;IS\$ & r(ii) &
						(ii),
						(iii),
						(iii)\\
						\hline
						$I_{1}$
						$I_{0}$ & ;IS\$ & s(4) &
						(ii),
						(iii),
						(iii)\\
						\hline
						$I_{4}$
						$I_{1}$
						$I_{0}$ & IS\$ & s(2) &
						(ii),
						(iii),
						(iii)\\
						\hline
						$I_{2}$
						$I_{4}$
						$I_{1}$
						$I_{0}$ & \$ & r(iii) &
						(iii),
						(ii),
						(iii),
						(iii)\\
						\hline
						$I_{5}$
						$I_{4}$
						$I_{1}$
						$I_{0}$ & \$ & r(ii) &
						(ii),
						(iii),
						(ii),
						(iii),
						(iii)\\
						\hline
						$I_{1}$
						$I_{0}$ & \$ & s(3) &
						(ii),
						(iii),
						(ii),
						(iii),
						(iii)\\
						\hline
						$I_{3}$
						$I_{1}$
						$I_{0}$ & \$ & $acc$ &
						(ii),
						(iii),
						(ii),
						(iii),
						(iii)\\
						\hline
					\end{tabular}
					\
					\\\\
					\\
					Podemos concluir que a pesar de que en ambos casos se genera la misma frase sin asociatividades `explicitas', al priorizar $shift$ nos genera una asociatividad `inplicita' a la izquierda, mientras que al priorizar $reduce$ nos genera una asociatividad `inplicita' a la derecha.
					\\\\
					%Parte (d)
					\item[(d)]
					Para comparar la eficiencia de ambas opciones, reconocimos frase de tamano $IS(;IS)^{2}$, $IS(;IS)^{3}$ , $IS(;IS)^{4}$ y $IS(;IS)^{5}$ para esrudiar sus comportamientos.
					\\\\
					Comparando el comportamiento de las pilas en ambas alternativas, vemos que dando prioridad al $shift$ el tamano maximo de la pila es $2*(n+1)$ para frases de la forma $IS(IS)^{n}$ mientras que para el $reduce$ el tamano maximo de la pila es 4 de manera constante.
					\\\\
					Por otro lado vemos que ambas alternativas cuestan $4(n + 1)$ pasos para aceptar la frase, luego, en cuestion de tiempo  son de orden $O(n)$.
					\\\\
					Podemos concluir que en cuestion de eficiencia la alternativa de $reduce$ es mas conveniente ya que cuesta la misma cantidad de tiempo, pero utiliza menos memoria en su pila.
					\end{enumerate}
									
			\end{enumerate}
\end{document}